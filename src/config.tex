\newif\ifkoma

%\usepackage[german]{fancyref} not used atm


\makeatletter
\@ifclassloaded{scrartcl}{\komatrue}{\komafalse}
\makeatother

\usepackage[usenames,dvipsnames,svgnames,table]{xcolor}

% Das wird in neueren Versionen von Pandoc benutzt:
\providecommand{\tightlist}{%
  \setlength{\itemsep}{0pt}\setlength{\parskip}{0pt}}

% Stellt \euro zur Verfügung:
\usepackage{eurosym} 

\usepackage{amssymb,amsmath}
\usepackage{ifxetex,ifluatex}
\ifxetex
  \usepackage{fontspec,xltxtra,xunicode}
  \defaultfontfeatures{Mapping=tex-text,Scale=MatchLowercase}
\else
  \ifluatex
    \usepackage{fontspec}
    \defaultfontfeatures{Mapping=tex-text,Scale=MatchLowercase}
    % Die 6 Vista-Fonts:
    %   1. Calibri:       Serifenlos, Ersatz für Verdana (war Ersatz für Arial)
    %   2. Cambria:       Ersatz für Times New Roman
    %   3. Candara:       (für informelle Schriftstücke)
    %   4. Consolas:      Nicht-Proportional-Schrift für Quelltexte
    %   5. Constantia:    Für Fließtexte für Print- und elektronische Medien
    %   6. Corbel:        Serifenlos, für die Bildschirmdarstellung, enthält Mediävalziffern
    \setsansfont{Calibri}
    \setromanfont{Cambria}
    %\setromanfont{Constantia}
    \setmonofont{Consolas}
  \else
    \usepackage[utf8]{inputenc}
    % UTF-8-Zeichen, die direkt im TeX-Quelltext stehen dürfen
    % --------------------------------------------------------
    % Doppelte Dollar-Zeichen für die math-Umgebung sind notwendig, wegen des
    % Pandoc-Expansionsmechanismus für Variablen.
    % Die Liste der Unicode-Zeichen, die über die Tastatur erreichbar sind,
    % findet man hier: 
    %   <url:/usr/share/X11/locale/en_US.UTF-8/Compose>
    % Der Compose-Key (= Multi_Key) ist ROLLEN (bei mh, einstellbar über
    % KDE-Einsgtellungen). 
    % Man drückt den Compose-Key, lässt ihn los und tippt dann zwei weitere
    % Zeichen ein.
    
    % €  Unicode Character 'EURO SIGN' (U+20AC), UTF-8 = E2 82 AC 
    % <url:http://www.fileformat.info/info/unicode/char/20aC/index.htm>
    \DeclareUnicodeCharacter{20AC}{\euro}

    % €  Unicode Character 'SECTION SIGN' (U+00A7), UTF-8 = c2 a7 (Paragraphenzeichen)
    \DeclareUnicodeCharacter{C2A7}{\S}

    % ←  Unicode Character 'LEFTWARDS ARROW' (U+2190),UTF-8 = E2 86 90, Compose< -  
    % vgl. <url:http://www.fileformat.info/info/unicode/char/2190/index.htm>
    \DeclareUnicodeCharacter{2190}{\mbox{$\leftarrow$}}

    % →  Unicode Character 'RIGHTWARDS ARROW' (U+2192), UTF-8 = E2 86 92, Compose- >:
    % (Das Zeichen wird in einigen Fonts wie ein Minus-Zeichen dargestellt. Das
    % ist ein Fehler im Font, der Zeichencode wird korrekt gebildet.)
    % vgl. <url:http://www.fileformat.info/info/unicode/char/2192/index.htm>
    \DeclareUnicodeCharacter{2192}{\mbox{$\rightarrow$}}

    %\usepackage{pxfonts}       % Palatino-likes fonts for TeX (breiter)
    %\usepackage{txfonts}        % Times-likes font for TeX (schmaler als Palatino) Julian disabled
    %\usepackage{lmodern}       % latin modern fonts
  \fi
\fi


\ifkoma
	%% Das ist ungetestet (2015-08-15):
	% Das ist der Font für "Abbildung 2":
	\addtokomafont{captionlabel}{\bfseries\small}
	% Das ist der Font für die eigentliche Beschreibung:
	\addtokomafont{caption}{\itshape\small}
	% Beschreibung nicht hängend unter Label setzen:
	\setcapindent{0em}
	% Das erzeugt »Abbildung 2.«, anstelle von »Abbildung 2:«.
	\renewcommand*{\captionformat}{.\ }
\fi

% if(tables)
\usepackage{ctable}
\usepackage{float} % provides the H option for float placement
% end(tables)

% if(url)
\usepackage[hyphens]{url}
% end(url)

%Funktioniert nicht gut:
% \renewcommand{\texttt}[1]{\color{red}#1\color{black}}

% Hier kommt z.B. Code von Pygments (automatisch):


 %%%%%%%%%%%%%%%%%%%%%%%%%%%%%%%%%%%%%%%%%%%%%%%
% Additions 
% define fbox border (for framing images e.g.)
\setlength\fboxrule{0.5pt}

\usepackage{subcaption}
\usepackage{float}
%\usepackage{subfig}

\makeatletter
\def\figcaption{%
	\refstepcounter{figure}%
	\@dblarg{\@caption{figure}}}
\makeatother


%\pagestyle{scrheadings}
%\usepackage[pdfspacing]{classicthesis}

\usepackage{enumitem}

%\clearscrheadfoot
%\ohead{\rightmark}
%\cfoot[\pagemark]{\pagemark}

\usepackage{translator} % macht aus "Acronyms" Akronyme
\deftranslation[to=German]{Acronyms}{Akronyme}

%Glossar, Abkürzungsverzeichnis, Symbolverzeichnis
\usepackage[nonumberlist, acronym, nomain, style=long]{glossaries}
%Glossar-Befehle anschalten
\makeglossaries
\renewcommand{\glsnamefont}[1]{\textbf{#1}}
	
%Diese Befehle sortieren die Einträge in den
%einzelnen Listen:
%makeindex -s datei.ist -t datei.alg -o datei.acr datei.acn
%makeindex -s datei.ist -t datei.glg -o datei.gls datei.glo
%makeindex -s datei.ist -t datei.slg -o datei.syi datei.syg

\newcommand*{\myglossaryindent}{0cm}

\newglossarystyle{longwithindent}{%
	\glossarystyle{long}%
	\renewenvironment{theglossary}%
	{\begin{longtable}[l]{@{\hspace{\myglossaryindent}}lp{\glsdescwidth}lp{\glspagelistwidth}@{}}}%
		{\end{longtable}}%
} 

\newacronym{RFC}{RFC}{Request for Comments}
\newacronym{CA}{CA}{Certification Authority}
\newacronym{RA}{RA}{Registration Authority}
\newacronym{VV}{VV}{Volksverschlüsselung}
\newacronym{PKI}{PKI}{Public Key Infrastructure}
\newacronym[plural={CRLs}, firstplural=Certificate Revocation Lists]{CRL}{CRL}{Certificate Revocation List}
%\acrodefplural{CRL}[CRLs]{Certificate Revocation Lists}
\newacronym{API}{API}{Application Programming Interface}
\newacronym{BC}{BC}{BouncyCastle}
\newacronym{UID}{UID}{User ID}
%\newacronym{UTF8}{UTF-8}{UTF-8 Zeichenkodierung}
\newacronym{SFTP}{SFTP}{Secure File Transfer Protocol}
\newacronym{HSM}{HSM}{Hardware Security Module}
\newacronym{JCA}{JCA}{Java Cryptography Architecture}
\newacronym{JCE}{JCE}{Java Cryptography Extension}
\newacronym{BLOB}{BLOB}{Binary Large Object}
\newacronym{PGP}{PGP}{Pretty Good Privacy}
\newacronym{GPG}{GPG}{GNU Privacy Guard}
\newacronym{WoT}{WoT}{Web of Trust}
\newacronym{E2E}{E2E}{Ende-zu-Ende}
\newacronym{OT}{OT}{Owner-Trust}
\newacronym{bnetza}{BNetzA}{Bundesnetzagentur}
\newacronym{IETF}{IETF}{Internet Engineering Task Force}
\newacronym{GPL}{GPL}{GNU Public License}
\newacronym{UTC}{UTC}{Coordinated Universal Time}
\newacronym{svn}{svn}{Subversion}
\newacronym{ECC}{ECC}{Elliptic-Curve-Cryptography}
\newacronym{BSI}{BSI}{Bundesamt für Sicherheit in der Informationstechnik}
\newacronym{CSP}{CSP}{Certification Service Provider}
\newacronym{PKC}{PKC}{Public Key Kryptographie}
\newacronym{ID}{ID}{Identifikationsnummer}
\newacronym{CMP}{CMP}{Certificate Management Protocol}
\newacronym[shortplural={GZHK},longplural={Größten Zusammenhangskomponente}]{GZHK}{GZHK}{Größte Zusammenhangskomponente}
\newacronym{JSON}{JSON}{Javascript Object Notation}
\newacronym{NIST}{NIST}{National Institute of Standards and Technology}
\newacronym{ER}{ER}{Entity-Relationship}
\newacronym{BNetzA}{BNetzA}{Bundesnetzagentur}



%\usepackage[square,sort,comma,numbers]{natbib}

\usepackage[backend=biber,style=alphabetic,natbib=true]{biblatex} % Use the bibtex backend with the authoryear citation style (which resembles APA)
\DefineBibliographyStrings{ngerman}{%
	urlseen={zuletzt besucht am:}% %%% not `visited on'
}
\setcounter{biburllcpenalty}{1000} % Zeilenumbrüche in Url bei jedem belibigen Buchstaben 

\usepackage[nameinlink]{cleveref}
\usepackage{nameref}

% Store \vref in a variable to enable redefining itself.
% Redefine \vref so the hyperlink is extended to the entire
% reference.
\let\vrefpointer\vref
\renewcommand{\vref}[1]{%
	\hyperref[#1]{\vrefpointer{#1}}%
}

% A custom command \smartref with the hyperlink extended to 
% the entire reference. 
% Plus some grammatic sugar (quote marks and comma)
\newcommand{\smartref}[1]{%
	\hyperref[#1]{\cref{#1}, \glqq\nameref{#1}\grqq, \vpageref{#1}}%
}

% define cref listings since its version is too old
\crefname{lstlisting}{}{} % if you fill "listing" and "listings" in there the noun will be doubled, to clear it...
\Crefname{lstlisting}{Listing}{Listings}

%%% fullref is evil! pages off, use vref

%fix for bug in biblatex http://tex.stackexchange.com/a/311428
\makeatletter
\def\blx@maxline{77}
\makeatother

\usepackage[autostyle]{csquotes}  

%Quellcode
\usepackage{listingsutf8}

%\usepackage[round, sort, compress, authoryear]{natbib}

%\usepackage[%
%style=ieee,%
%%   style=numeric,
%isbn=true,%
%doi=false,%
%sorting=none,%
%url=true,%
%defernumbers=true,%
%bibencoding=utf8,%
%backend=biber%
%]{biblatex}

%Den Punkt am Ende jeder Beschreibung deaktivieren
\renewcommand*{\glspostdescription}{}

%%%%%%%%%%%%%%%%%%%% Quelltext Element - Style-File %%%%%%%%%%%%%%%%%%%%%%%
\usepackage[varqu]{zi4}
\definecolor{eclipse-red}{RGB}{127,0,85}
\definecolor{eclipse-blue}{RGB}{0,0,192} %for fields
\definecolor{eclipse-strings}{RGB}{42,0,255}
\definecolor{eclipse-green}{RGB}{63,127,95}
\definecolor{eclipse-lightblue}{RGB}{127,159,191} %for Javadoc Tags
\definecolor{eclipse-gray}{RGB}{100,100,100} %for annotations
\definecolor{eclipse-JavadocHTML}{RGB}{127,127,159} %for Javadoc HTML
\definecolor{eclipse-JavadocLinks}{RGB}{63,63,191}
\definecolor{eclipse-Javadoc}{RGB}{63,95,191}
\lstdefinestyle{eclipse-java}{%
	basicstyle=\ttfamily\small,%\fontfamily{zi4}\fontsize{10pt}{10pt}\selectfont,%
	tabsize=2,%
	breakautoindent=true,%
	breaklines,%
	postbreak=\space\space\space,%
	breakindent=0pt,%
	keywordstyle=\color{eclipse-red}\bfseries,%
	stringstyle=\color{eclipse-strings},%
	commentstyle=\color{eclipse-green},
	showstringspaces=false,
	morecomment=*[s][\color{eclipse-Javadoc}]{/**}{*/},
	morecomment=*[l][commentstyle]{//},
	language=Java,morekeywords={enum},
	emph={field,field2},emphstyle={\color{eclipse-blue}},
	emph=[2]{RED,GREEN,BLUE,staticField,inheritedField},emphstyle=[2]{\color{eclipse-blue}\itshape},
	emph=[3]{staticMethod},emphstyle=[3]\itshape,
	emph={[4]TASK},emphstyle={[4]\color{eclipse-lightblue}\bfseries},
	alsoletter={@},
	emph=[5]{@author,@deprecated},emphstyle=[5]{\color{eclipse-lightblue}\bfseries},
	moredelim=[l][\color{eclipse-gray}]{@},
	moredelim=*[s][\color{eclipse-Javadoc}]{/**}{*/},
	moredelim=[s][\color{eclipse-JavadocLinks}]{@link}{*},
	extendedchars=true,%
	literate={Ö}{{\"O}}1 {Ä}{{\"A}}1 {Ü}{{\"U}}1 {ß}{{\ss}}1 {ü}{{\"u}}1
	{ä}{{\"a}}1 {ö}{{\"o}}1}

\lstset{
	breakatwhitespace=false,
	breaklines=true,
	frame=single,
	numbers=left,
	escapeinside={(*}{*)},
	basicstyle=\footnotesize\ttfamily,
	deletekeywords={this},
%	columns=fixed
}


\newcommand\highlight[1]{\colorbox{yellow}{#1}}


%%% BEGIN grey formal cite block
% for adjustwidth environment
\usepackage[strict]{changepage}

% for formal definitions
\usepackage{framed}

% environment derived from framed.sty: see leftbar environment definition
\definecolor{formalshade}{rgb}{0.95,0.95,1}

\newenvironment{formal}{%
	\def\FrameCommand{%
		\hspace{1pt}%
		{\color{DarkBlue}\vrule width 2pt}%
		{\color{formalshade}\vrule width 4pt}%
		\colorbox{formalshade}%
	}%
	\MakeFramed{\advance\hsize-\width\FrameRestore}%
	\noindent\hspace{-4.55pt}% disable indenting first paragraph
	\begin{adjustwidth}{}{7pt}%
		\vspace{2pt}\vspace{2pt}%
	}
	{%
		\vspace{2pt}\end{adjustwidth}\endMakeFramed%
}
%%%% END grey formal cite block

 %%%%%%%%%%%%%%%%%%%%%%%%%%%%%%%%%%%%%%%%%%%%%%%%%%%%%%%%%%%%%%%%%%%%%%%%%%%%%%%%%%%%%%%%%%%%%%%%%%


% Die Variable 'cs' wird offensichtlich von pandoc gesetzt, wenn Bilder
% eingebunden werden.
% In diesem Fall wird das Macro '\includegraphics' umdefiniert, um festzulegen,
% dass alle Bilder auf die Seitenbreite vergrößert/verkleinert werden. 
\usepackage{graphicx}

\let\Oldincludegraphics\includegraphics
% \let\includegraphicsoriginal\includegraphics
% if(graphics)
% We will generate all images so they have a width \maxwidth. This means
% that they will get their normal width if they fit onto the page, but
% are scaled down if they would overflow the margins.
\makeatletter
\def\maxwidth{\ifdim\Gin@nat@width>\linewidth\linewidth
\else\Gin@nat@width\fi}
\makeatother
\renewcommand{\includegraphics}[1]{\Oldincludegraphics[width=1.0\maxwidth]{#1}}
% end(pictures)
% \begin{figure} .. \end{figure} wird umdefiniert,
% um Bilder immer vor Ort einzufügen, nicht als gleitende Objekte.
\let\Oldfigure=\figure
\def\figure[#1]{\Oldfigure[h!]}
% end(graphics)

%\hypersetup{breaklinks=true, bookmarks=true}
            
%\hypersetup{breaklinks=true, pdf={0 0 0}, urlcolor=blue}
            % Das unterdrückt den Rahmen um Links,  
            % indem die Farbe auf weiß gesetzt wird:
            % allbordercolors=1 1 1,
            % urlbordercolor=1 1 1,


\setlength{\parindent}{0pt}
\setlength{\parskip}{6pt plus 2pt minus 1pt}
\setlength{\emergencystretch}{3em}  % prevent overfull lines


% end(lang)

% for(header-includes)
% end(header-includes)


% --------------------------------------------
% Es folgt eine Kopie von colab.sty mit Kommandos
% zur Unterstützung des colab-Filters.
% Die Kommandos werden unterdrückt, wenn eine Datei `mycolab.sty` im 
% aktuellen Verzeichnis vorhanden ist.
% Es ist zu beachten, dass in Pandoc auf ein Dollarzeichen ein Buchstabe folgen muss.
% Ein Mathematikbefehl wie Dollar \Box Dollar führt daher zu einem Fehler.
% Man verhindert ihn, indem man in TeX \( .. \) statt Dollar .. Dollar schreibt.

% Innerhalb einer \IfFileExists-Anweisung darf man offensichtlich keine
% Kommandows mit #1 definieren. Daher wird ein neues Boolean eingeführt,
% innerhalb dessen alles erlaubt ist.
%\newif\ifmycolabExists
%\IfFileExists{./mycolab.sty}{\mycolabExiststrue}{\mycolabExistsfalse}

%\ifmycolabExists
%  \input{./mycolab.sty}

  % mdframed hat Probleme mit dem Seitenumbruch (2016-01-31).
  %\usepackage[framemethod=default]{mdframed} 
  \usepackage{tcolorbox} 

  \definecolor{lightyellow}{rgb}{1,1,.5} 
  \DefineNamedColor{named}{ForestGreen}{cmyk}{0.91,0,0.88,0.12}

  % 1. Inline-Kommentare (~~ ... ~~):
  \newcommand{\inlinecomment}[1]{{\textsf{\color{black}\colorbox{lightyellow}{\(\langle\)#1\(\rangle\)}}}}

  % 2. Lange Kommentare (<div class="comment"> ... </div>):
  %
  % Das Environment comment wird im Paket verbatim.sty definiert,
  % welches von tcolorbox.sty importiert wird,
  % daher kann die Klasse comment nicht auf das gleichnamige Environment
  % abgebildet werden.
  %
  % Das funktioniert mit dem Paket mdframed.
  %\newmdenv[roundcorner=5pt,backgroundcolor=lightyellow]{comment}
  % Das funktioniert mit dem Paket tcolorbox:
  \renewenvironment{comment}{\begin{tcolorbox}[colback=lightyellow]}{\end{tcolorbox}}

  % 3. Todo (<div class="todo"> ... </div>):
  \newenvironment{todo}{%
  \par
  \makebox[0cm][r]{\textbf{\textcolor{red}{TODO~}}}%
  \small}{\textcolor{red}{\(\Box\)}\par}

  % 4.Done (<div class="done"> ... </div>):
  \newenvironment{done}{%
  \par
  \makebox[0cm][r]{\textbf{\textcolor{ForestGreen}{DONE~}}}%
  \small}{\textcolor{ForestGreen}{\(\Box\)}\par}
%\fi

% Ende von colab.sty.
% --------------------------------------------

% Um Linien am Rand zur Erkennung von Zitaten zu erzeugen:
% \usepackage{changebar}  
% \changebarsep-5mm

% ipad beachten:
%     Die Variable ipad wird z.B. von dem Script
%     <url:/home/herfert/p/herfert/s/lang/zsh/pandoc2pdf-for-vim.zsh>
%     gesetzt, falls im Verzeichnis des pandoc-Dokuments eine Datei
%     ipadyes (beliebigen Inhalts) existiert.

\usepackage{longtable}
\ifkoma
	\usepackage[automark]{scrpage2}
\fi

\setlength{\aboverulesep}{0pt}
\setlength{\belowrulesep}{0pt}
\renewcommand*{\arraystretch}{1.2}%

%\usepackage{boolexpr,pdftexcmds,trace}
%\usepackage{boolexpr}



% Das soll für gute Trennungen sorgen und den Einsatz von
% \sloppy  überflüssig machen.
% Vorteil gegenüber 
%   \slopyy (= lockerer Umgang mit Trennungen)
%  und
%   \fussy (= penibler Umgang mit Trennungen):
% Die  Version unten arbeitet harmonischer, reist weniger Löcher in das Druckbild. 
% Quelle: <url:http://www.dante.de/CTAN/info/german/l2tabu/l2tabu.pdf>
% lokal:  <url:file:///usr/share/doc/texlive-doc-de/german/l2tabu/l2tabu.pdf>
\tolerance 1414
\hbadness 1414
\emergencystretch 1.5em
\hfuzz 0.3pt
\widowpenalty=10000
\vfuzz \hfuzz
\raggedbottom
 

% Quelle für String-Vergleich:
%   <url:http://tex.stackexchange.com/questions/29133/how-to-create-switch-structure-comparing-strings-in-latex>
\long\def\isequal#1#2{\pdfstrcmp{#1}{#2}}

\ifkoma
% Abschnittsbezeichnung im Seitenkopf:
\pagestyle{scrheadings}

% Addition : force right-only pagenumbers
\clearscrheadfoot
\ohead{\raggedleft\rightmark}
\cfoot[\pagemark]{\raggedleft\pagemark}
%% end addition
\fi

% Hyperlinks sollen immer als Fußnoten gedruckt werden, damit alle Informationen
% auch im gedruckten Text vorhanden sind:
%\newcommand{\originalhrefcommand}{}
%\let\originalhrefcommand=\href
%\renewcommand{\href}[2]{%
%\originalhrefcommand{#1}{#2}%
%\footnote{\url{#1}}%
%}


% Wenn man eine Nummerierung des Literaturverzeichnisses vermeiden will,
% benutzt man: 
% # Literatur {.unnumbered}
% oder das äquivalente
% # Literatur {-}

% Nach \LaTeX soll ein expliziter Leerraum erzeugt werden,
% erspart \LaTeX{}.
\let\origLaTeX=\LaTeX
\renewcommand{\LaTeX}{\origLaTeX~}


\usepackage{soul} % for strikethrough

% my@and sieht nach, ob nach my@and noch ein Autor folgt.
% Falls ja, wird es zu \and expandiert.
% Falls ein, wird es zu nichts expandiert.
\usepackage{ifthen,xifthen}
\makeatletter
\newlength{\m@length}

\def\my@and#1{%
\settowidth{\m@length}{#1}%
\ifthenelse{\lengthtest{\m@length = 0pt}}{}{\and #1}%
}

% Angabe mehrerer Autoren mit yaml:
% author:
% - Luke Skywalker
% - Darth Vader
% Das Problem mit der for-Schleife ist, 
% dass am Ende ein my@and übrigbleibt,
% das erkennen muss, dass es eigentlich nicht gebraucht wird.
\makeatother


% Erzeugt a), b) usw. für Aufzählungen der Stufe 2:
% (funktioniert nicht, weil Labels im .pan explizit angegeben werden.)
%\renewcommand{\labelenumii}{\alph{enumii})}

% Trennungen haben eigentlich wenig mit dem Pandoc-Stil zu tun,
% aber das ist ein pragmatischer Ort.
\hyphenation{Omni-Cloud}

% Eine Environment-Definition wie \begin{quote} .. \end{quote}
% wird intern in die Befehle
%  \quote und \endquote
% übersetzt. Diese beiden muss man mit \let retten,
% um später auf sie zugreifen zu können.
%
% Vermutlich sind die \newcommand nur notwendig,
% um Doppeldefinition zu entdecken.
\newcommand{\origquote}{}
\newcommand{\origendquote}{}
\let\origquote=\quote
\let\origendquote=\endquote
\renewenvironment{quote}%
{\origquote\itshape}%
{\origendquote}

%{\cbstart\origquote\itshape}%
%{\origendquote\cbend}


%\setlength{\textheight}{22cm}

% Define some commands to keep the formatting separated from the content 
\newcommand{\keyword}[1]{\textbf{#1}}
\newcommand{\tabhead}[1]{\textbf{#1}}
\newcommand{\code}[1]{\texttt{#1}}
\newcommand{\file}[1]{\texttt{\bfseries#1}}
\newcommand{\option}[1]{\texttt{\itshape#1}}


% reconfigure representation of chapters (space befor and after)
\usepackage{titlesec}
\titleformat{\chapter}[display]   
{\normalfont\huge\bfseries}{\chaptertitlename\ \thechapter}{18pt}{\Huge}   
\titlespacing*{\chapter}{0pt}{-50pt}{5pt}

\newcommand{\centereddots}{\noindent\makebox[\linewidth-1em][c]{\textbf{…}}}

% Tabellen
\usepackage{multirow}
\usepackage[usestackEOL]{stackengine}
\usepackage{tabularx}
\usepackage[flushleft]{threeparttable}
\usepackage{tablefootnote}

\vrefwarning

% define round box around text
\usepackage{tcolorbox}
\definecolor{bluegrey}{rgb}{0.122, 0.435, 0.698}
\newtcbox{\quickbox}{nobeforeafter,colframe=bluegrey,colback=bluegrey!10!white,boxrule=0.5pt,arc=4pt,boxsep=0pt,left=3pt,right=3pt,top=3pt,bottom=3pt,tcbox raise base}


% define listing for varioref since it doesn't recognize it else
\AtBeginDocument{\labelformat{lstlisting}{Listing~#1}}


%This removes the space between 2 chapters only when they are just next to each other.
% from http://tex.stackexchange.com/a/89774/93829
\usepackage[titles]{tocloft}
\makeatletter
\newskip\old@cftbeforechapskip
\old@cftbeforechapskip\cftbeforechapskip
\let\old@l@chapter\l@chapter
\let\old@l@section\l@section
\def\l@chapter#1#2{\old@l@chapter{#1}{#2}\cftbeforechapskip3pt}% set your value here
\def\l@section#1#2{\old@l@section{#1}{#2}\cftbeforechapskip\old@cftbeforechapskip}
\makeatother

\usepackage{changepage} % for indented  text